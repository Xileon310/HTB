\documentclass[a4paper]{article} % Template formating

\usepackage[utf8]{inputenc}
\usepackage[spanish]{babel}
\usepackage[margin=2cm, top=2cm, includefoot]{geometry}
\usepackage{graphicx} % Insert imgs
\usepackage[table,xcdraw]{xcolor} % Detect colours
\usepackage[most]{tcolorbox} % Insert boxes
\usepackage{fancyhdr} % Define page style
\usepackage[hidelinks]{hyperref} % Hyperlinks management
\usepackage{parskip} % Delete tab indent
\usepackage[figurename=Image]{caption} % Change caption name
\usepackage{smartdiagram} % Insert diagrams
\usepackage{zed-csp} % Insert schemas

% Colour declaration
\definecolor{greenCoverPage}{HTML}{69A84F}

% Variable declaration
\newcommand{\logoCoverPage}{../img/hackTheBox.png}
\newcommand{\machineName}{Previse} % Machine name
\newcommand{\logoMachine}{../img/easy/previse/Previse.png}
\newcommand{\startDate}{November 22, 2021}

% Utils
\addto\captionsspanish{\renewcommand{\contentsname}{Index}} % Change Index Formatting
\setlength{\headheight}{40.2pt}
\pagestyle{fancy}
\fancyhf{}
\lhead{\includegraphics[width=5cm]{\logoCoverPage}}

%\rhead{\includegraphics[height=1.5cm]{\logoMachine}}

\renewcommand{\headrulewidth}{3pt}
\renewcommand{\headrule}{\hbox to \headwidth{\color{greenCoverPage}
    \leaders\hrule height \headrulewidth\hfill}}
\renewcommand{\lstlistingname}{Code} % Change listing name

% Code listing
\definecolor{codegreen}{rgb}{0,0.6,0}
\definecolor{codegray}{rgb}{0.5,0.5,0.5}
\definecolor{codepurple}{rgb}{0.58,0,0.82}
\definecolor{backcolour}{rgb}{0.95,0.95,0.92}

\lstdefinestyle{mystyle}{
    backgroundcolor=\color{backcolour},   
    commentstyle=\color{codegreen},
    keywordstyle=\color{magenta},
    numberstyle=\tiny\color{codegray},
    stringstyle=\color{codepurple},
    basicstyle=\ttfamily\footnotesize,
    breakatwhitespace=false,         
    breaklines=true,                 
    captionpos=b,                    
    keepspaces=true,                 
    numbers=left,                    
    numbersep=5pt,                  
    showspaces=false,                
    showstringspaces=false,
    showtabs=false,                  
    tabsize=2
}

\lstset{style=mystyle}

% Main

\begin{document}
    \cfoot{\thepage}

    % -------------------------------------------------------------------------
    % Cover Page
    \begin{titlepage}
    \centering
    \includegraphics[width=0.6\textwidth]{\logoCoverPage}\par
    \vspace{0.5cm}
    {\scshape\LARGE \textbf{Writeup}\par}
    \vspace{0.5cm}
    {\Huge\bfseries\textcolor{greenCoverPage}{Machine \machineName}\par}
    \vspace{0.5cm}
    \includegraphics[width=\textwidth, height=10cm, keepaspectratio]
        {\logoMachine}\par
    \vspace{2cm}
    \begin{tcolorbox}[colback=red!5!white, colframe=red!75!black]
        \centering
        This writeup is public and it can be found on my github: 
            \href{https://github.com:Xileon310}{\textbf{\color{blue}Xileon310}}
    \end{tcolorbox}
    \vfill
    {\large\startDate\par}
    \vfill
    \end{titlepage}

    % -------------------------------------------------------------------------
    % TOC (Table Of Contents) Start
    \clearpage
    \tableofcontents
    \clearpage
    % -------------------------------------------------------------------------

    \section{Tools}
    We will use the following tools:
    \begin{itemize}
        \item Nmap
        \item Burpsuite
        \item Gobuster / Dirbuster
        \item Hashcat
        \item Netcat
    \end{itemize}
        
    \section{Getting User Flag}
    We first start with a NMAP Scan.

    \begin{center}
        \includegraphics[width=0.8\textwidth]{../img/easy/previse/nmap_scan.png}
    \end{center}

    Port 80 and 22 is open, so we will start using gobuster to bruteforce the
    server and know the available routes.\par

    We were accessing to the different routes, in special to nav.php.

    \begin{center}
        \includegraphics[width=0.4\textwidth]{../img/easy/previse/navphp.png}
    \end{center}

    After this, we got lots of menus, but when I tried to open each of them,
    one by one, I kept getting redirected to login.php.\par
    I will intercept the response from server when I try to visit CREATE ACCOUNT
    route. I discovered that we were able to visit the page but immediately 
    redirected back to the login page.
    We intercepted the response and changed the 302 code to 200 OK.

    \begin{center}
        \includegraphics[width=0.8\textwidth]{../img/easy/previse/burpsuite.png}
    \end{center}

    This worked fine, we accesed to the add new account panel and added an user.

    \begin{center}
        \includegraphics[width=0.8\textwidth]{../img/easy/previse/account_panel.png}
    \end{center}

    Now, we can navigate around the website like a normal user. We investigated
    the web page carefully and downloaded the backup.zip in Files section.
    I checked all the files, especially the config.php, which showed us mysql
    credentials.\par

    Also, we could see that in logs.php uses exec(), and this function is 
    dangerous. It could give us a code execution if our input is not handled
    properly. We will use it to input malicious code via Burpsuite intercepting
    the request in log data section.

    \begin{center}
        \includegraphics[width=0.8\textwidth]
            {../img/easy/previse/malicious_burpsuite.png}
    \end{center}

    Yes, using netcat we got a shell from the server.

    \begin{center}
        \includegraphics[width=0.8\textwidth]{../img/easy/previse/ncshell.png}
    \end{center}

    Remember that we got mysql credentials some minutes ago, so log in mysql
    and lets see what we can get.

    \begin{center}
        \includegraphics[width=0.8\textwidth]{../img/easy/previse/mysql.png}
    \end{center}

    We got a md5crypt, so I will copy it and bruteforce with hashcat.

    \begin{center}
        \includegraphics[width=0.8\textwidth]{../img/easy/previse/hashcat.png}
    \end{center}

    Now, we have the user \textbf{m4lwhere} and the password \textbf{*******}.
    Log in via ssh and grab the flag.

    \section{Getting Root Flag}
    We use \textbf{sudo -l} to know which program could be executed as root.

    \begin{center}
        \includegraphics[width=0.8\textwidth]{../img/easy/previse/sudol.png}
    \end{center}

    This script uses \textbf{gzip} command indirectly. This makes the script
    vulnerables to \$PATH manipulation. We will use it to gain the root shell.
    \par
    We will make a fake binary with the name \textbf{gzip} in the /tmp directory,
    also, we have to add the /tmp directory to the \$PATH variable.
    \par
    This binary should contain a way to scalate privileges, in my case, I used a
    reverse shell with netcat as we did previously.

    \section{Thoughts}
    This machine is an easy machine and very interesting. It teaches us to check
    and modify the web response. Also, introduces the hashes in a very shallow
    shape, which is a good start for anyone.
\end{document}
